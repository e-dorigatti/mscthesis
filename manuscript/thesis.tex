\documentclass[12pt,twoside]{article}
\usepackage{graphicx}
\usepackage[a4paper,inner=36mm,outer=18mm,top=30mm,]{geometry}
%\usepackage[doublespacing]{setspace} 
\usepackage{mathptmx}
\usepackage{titling}
\usepackage{lipsum}


\begin{document}

\title{Thesis}
\author{Emilio Dorigatti}

\maketitle

\section{Summary}
todo

\section{Abstract}
todo

\section{Introduction}
todo

\section{Background}
This section introduces the basic concepts the reader should be qualitatively familiar with, in order to understand the content of this thesis. It is assumed readers are already knowledgeable of simple mathematical concepts, such as calculus and linear algebra. Readers acquainted with the material should feel free to skip this chapter.

\subsection{Fluid Dynamics and the Boundary Layer}
Fluid dynamics is the discipline that studies the flow of fluids; it has several branches that study different fluids, such as aerodynamics (the study of air motion) and hydrodynamics (the study of water motion). These disciplines are routinely applied when designing cars, airplanes, ships, pipelines, etc.

\subsubsection{Laminar and Turbulent flow}
todo

\subsubsection{The Boundary Layer}
In the context of fluid dynamics, the boundary layer studies the behavior of a fluid when it is flowing close to a solid surface. Fluids are characterized by their viscosity, which describes, intuitively, how much the molecules of the fluid tend to stick together and resist motion by generating drag. For example, water has low viscosity, and honey has high viscosity. Imagine a laminar flow close to a solid surface; because of viscosity, the molecules flowing near the surface move slower, and, in the limit, the velocity of the molecules in direct contact with the surface is 0 (this is called the \emph{non-slip condition}). Thus, the velocity of the fluid increases smoothly, continuously and monotonously with the distance from the solid, until it reaches the \emph{free-flow} velocity, after which it stays constant. The region close to the surface, where the fluid moves slower, is called the \emph{boundary layer}, and is the region where the viscosity of the fluid influences its motion. Its height $\delta$ can be defined when the local velocity surpasses a certain threshold, such as 99\% of the free-flow velocity.

The variation of velocity with distance from the surface, $\partial\overline{u}/\partial z$, is called \emph{shear}, and, together with viscosity, determines the materialization of turbulence in the flow. Every layer of fluid is squeezed between a faster moving layer above and a slower moving below; in high shear conditions, this causes high stress on the particles, and prevents them from moving orderly, thus leading to turbulent motion. Viscosity and the no-slip condition prevent this phenomenon to arise in a region very close to the solid surface, called the \emph{laminar sub-layer}, where we still find laminar motion.

The strength of the turbulence is proportional to $u_{rms}=(\overline{u'^2})^{1/2}$, which is, in turn, proportional to the shear. Again, because of the no-slip condition, $u_{rms}$ is zero at $z=0$, increases in the laminar sub-layer, and decreases to 0 at the end of the boundary layer, assuming laminar flow outside of it. Higher free-stream velocity generates higher shear, more turbulence, and a thinner laminar sub-layer. The strength of turbulence can be written in units of velocity, resulting in the \emph{friction velocity}, computed as $u_*=(\tau/\rho)^{1/2}=(\nu\cdot\partial\overline{u}/\partial z)^{1/2}$, where $\tau$ is the shear stress, $\rho$ is the density of the fluid, $\nu=\mu/\rho$ is the kinematic viscosity, and $\mu$ the dynamic viscosity. Therefore, the friction velocity increases with shear and viscosity, and decreases with density; it is proportional to the free-stream velocity and the turbulence strength, and inversely proportional to the height of the laminar sub-layer.

The mean velocity $\overline{u}$ increases linearly within the laminar sub-layer, then logarithmically until the end of the boundary layer, thus the shear decreases further away from the surface. In the logarithmic sub-layer, the velocity is computed as $\overline{u}(z)=u_*(\log z - \log z_0)/\kappa$, where $z_0$ is the characteristic roughness of the surface, and $\kappa$ is the von Karman's constant, whose value is around 0.4 [citation needed]. The characteristic roughness depends on the texture of the surface, and its relationship with the height $\delta_s$ of the laminar sub-layer; if the roughness scale is smaller than $\delta_s$, the logarithmic velocity profile is not affected by the texture, because the laminar sub-layer completely covers the variations on the surface, and we have the so-called smooth turbulent flow. If, on the contrary, the bumps in the surface are larger than $\delta_s$, the laminar sub-layer follow the profile of the surface, and the logarithmic velocity profile is altered depending on the texture, a regime called rough turbulent flow.


\subsection{The Atmospheric Boundary Layer}
\lipsum

\subsubsection{The Turbulence Kinetic Energy Budged}
\lipsum

\subsubsection{Surface Fluxes}
\lipsum

\subsubsection{Monin-Obukhov Similarity Theory}
\lipsum

\end{document}